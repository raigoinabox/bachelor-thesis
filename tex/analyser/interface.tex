This project is run through the Command Line Interface. The user will find the
file to run in the core folder. The file is called \verb;analyse-wiki.py; and it
needs to be run with the command \verb;./analyse-wiki.py;. The default behaviour
of the script is to retrain the machine learning algorithm and filter the good
pages from all of Vikipeedia. The command can also take one argument. If the
argument is \verb;train;, then the command will only retrain the machine
learning algorithm. If the argument is \verb;find;, the command will only search
for good pages using the result of the last retraining. 

% TODO, tee nii, et ma lihtsalt prindin ükshaaval kõikide lehtede url'id, mis ma
% leian heana, terminalile ja ei ole vaja salvestada pickle formaadis faili.
% Kuigi, räägime sellest veel.

After the bot has found all of the good pages, it will then write the list of
good pages in a file named \verb;good_pages.pkl; using the Python library
\verb;pickle; and print it out on the terminal.