\emph{The author used Arch Linux of \today to make this program and hasn't
tested installing and running it on other systems. Proceed on your own caution.}

This project requires the Python 2 interpreter on the system. As of \today
PyWikiBot does not support Python 3 and therefore this paper's code was also not
written in Python 3. The Python library numpy is also necessary.
One can install them with the terminal command 
\verb;sudo pacman -S --needed python2 python2-numpy;.

A user-config.py configuration file must also exist. To generate it, there is a
script in the pywikibot folder. It's named \verb;generate_user_files.py; and and
it must be run with the Python interpreter. This is a sample installation
process:
\begin{verbatim}
raigo@archofraigo ~/git/wiki-analyse-bot/core (git)-[master] % python2 generate_user_files.py                                                                               :(

Your default user directory is "/home/raigo/.pywikibot"
How to proceed? ([K]eep [c]hange) 
Do you want to copy user files from an existing pywikipedia installation? n
Create user-config.py file? Required for running bots ([y]es, [N]o) y
1: anarchopedia
2: battlestarwiki
[...]
26: wikinews
27: wikipedia
28: wikiquote
[...]
33: wiktionary
34: wowwiki
Select family of sites we are working on, just enter the number not name (default: wikipedia): 
This is the list of known language(s):
ab ace [...] es et eu [...] zh-yue zu
The language code of the site we're working on (default: 'en'): et
Username (et wikipedia): AnalyseBot
Which variant of user_config.py:
[S]mall or [E]xtended (with further information)? S
Do you want to add any other projects? (y/N)
'/home/raigo/.pywikibot/user-config.py' written.
Create user-fixes.py file? Optional and for advanced users ([y]es, [N]o) 
\end{verbatim}
Questions without answers use the default answer by pressing Enter.

The \verb;analyse-wiki.py; file needs to have the right to execute. Otherwise
one must use the python interpreter to run it. This can be achieved with the
command \verb;chmod u+x analyse-wiki.py;
