\subsubsection{Functional programming}

Functional programming languages are designed around functions. Programming
functions are similar to mathematical functions, that it has inputs as
parameters and returns a value as output. A function should always return the
same output with the same inputs and the inputs should not be changed inside the
function. This leads to a particularly stateless form of programming.

Due to the stateless form of this style, functional programming languages
usually support immutable datastuctures. Instead of updating a datastructure, it
is copied with the new values replaced and the new datastucture is returned.
First-class functions and dynamic evaluation of functions are also
supported.\cite{website:persistent-struct}

Python functions are very similar to Python procedures:
\begin{verbatim}
>>> def fib2(n): # return Fibonacci series up to n
...     """Return a list containing the Fibonacci series
...     up to n."""
...     result = []
...     a, b = 0, 1
...     while a < n:
...         result.append(a)    # see below
...         a, b = b, a+b
...     return result
...
>>> f100 = fib2(100)    # call it
>>> f100                # write the result
[0, 1, 1, 2, 3, 5, 8, 13, 21, 34, 55, 89]
\end{verbatim}
What has changed is that now the results are being collected into a list and
then the list is returned with the \verb;return;
keyword.\cite[4.6. Defining Functions]{website:python-functions}

A more functional way of programming would be this:
\begin{verbatim}
>>> def fib2(n): # return Fibonacci series up to n
...     """Return a list containing the Fibonacci series
        up to n."""
...     a, b = 0, 1
...     while a < n:
...         yield a    # see below
...         a, b = b, a+b
... 
>>> list(fib2(100))
[0, 1, 1, 2, 3, 5, 8, 13, 21, 34, 55, 89]
\end{verbatim}
In this example we don't change the list to collect the results, but instead
\verb;yield; the result. When the function is called, instead of returning a
final result, it will return a generator. A generator is a simple data
structure, that is iterable through only once. The \verb;list; function takes an
iterable and returns a list with the iterable's elements.

A \verb;yield; created generator will evaluate the function until the first
\verb;yield; keyword, return the result and pause the function. When the next
element is asked for, the function is continued and return the result of the
next \verb;yield;. When the end of the function is reached, the generator stops.

\begin{quote}
``Small anonymous functions can be created with the \verb;lambda; keyword. This
function returns the sum of its two arguments: \verb;lambda a, b: a+b;. Lambda
functions can be used wherever function objects are required. They are syntactically restricted to a single expression. Semantically, they are just syntactic sugar for a normal function definition. Like nested function definitions, lambda functions can reference variables from the containing scope:''\cite[4.7.5. Lambda Expressions]{website:python-functions}
\end{quote}
\begin{verbatim}
>>> def make_incrementor(n):
...     return lambda x: x + n
...
>>> f = make_incrementor(42)
>>> f(0)
42
>>> f(1)
43
\end{verbatim}

Python has support for lexical closures, which gives the function a strong
reference to the namespace. This excludes the possibility that the namespace will be garbage
collected while the function is still in memory. The function has the guarantee
that the non-local values will exist even after the enclosing context is deleted
or garbage collected.\cite{website:python-closures}

Since other functional languages have them, Python also has functions
\verb;map;, \verb;filter; and \verb;reduce;. \verb;map; applies a function
to every item in an iterable and returns a new list with the results of the
function. \verb;filter; applies a function to every item in an iterable and
returns a new list with the items where the function returned \verb;True;. And \verb;reduce; applies a function for every item in an iterable and returns the value of the last evaluation. In this case the function must have two parameters and the first parameter holds the value from the last evaluation of the function.

A way to avoid using \verb;map; and \verb;filter; is by generator expressions
and list comprehensions. List comprehensions are a syntactic sugar to easily
create lists. Unlike \verb;map;, list comprehensions don't need to be supplied a
function, but can also use arbitrary expressions. A sample list comprehension
is: \verb;[2 * item for item in iterable if item % 2 == 0];. This expression
returns a filtered list where each element of iterable is multiplied with 2. The
returned list contains only elements that were even before. Therefore the syntax
is
\verb;[expression for expr1 in sequence1 if condition1 for expr2 in sequence2 if condition2 ...];.
The commands are nested with the right being inside the left and the first expression being the returned innermost expression. Generator expressions are syntactically same, except normal brackets instead of square brackets are used and they create a generator instead of a
list.

Functional programming requires immutable data structures. Fot that python has
tuples and named tuples. A tuple is an immutable data structure, that allows
packing different data together and afterwards unpack them or access them with
an index. Tuples are immutable, meaning they can't be changed.
\begin{verbatim}
>>> t = 12345, 54321, 'hello!'
>>> t[0]
12345
>>> t
(12345, 54321, 'hello!')
>>> # Tuples may be nested:
... u = t, (1, 2, 3, 4, 5)
>>> u
((12345, 54321, 'hello!'), (1, 2, 3, 4, 5))
>>> # Tuples are immutable:
... t[0] = 88888
Traceback (most recent call last):
  File "<stdin>", line 1, in <module>
TypeError: 'tuple' object does not support item assignment
>>> # but they can contain mutable objects:
... v = ([1, 2, 3], [3, 2, 1])
>>> v
([1, 2, 3], [3, 2, 1])
\end{verbatim}
\verb;namedtuple; is a function in the \verb;collections; module that returns a
class that inherits from the tuple class. Named tuples have all the properties
of a tuple and add on a couple of things:
\begin{enumerate}
  \item The named tuple has a name and a fitting string representation.
  \item Every field in a named tuple is registered with a name instead of an
  index.
\begin{enumerate}
  \item You can initialise the named tuple with the named tuples.
  \item You can access the fields of the named tuple by their names.
  \item You can't construct a named tuple with more or less elements than it
  expects.
\end{enumerate}
\end{enumerate}
\begin{verbatim}
>>> Point = namedtuple('Point', ['x', 'y'], verbose=True)
class Point(tuple):
    'Point(x, y)'

    __slots__ = ()

    _fields = ('x', 'y')

    def __new__(_cls, x, y):
        'Create a new instance of Point(x, y)'
        return _tuple.__new__(_cls, (x, y))

    @classmethod
    def _make(cls, iterable, new=tuple.__new__, len=len):
        'Make a new Point object from a sequence or iterable'
        result = new(cls, iterable)
        if len(result) != 2:
            raise TypeError('Expected 2 arguments,
                            got %d' % len(result))
        return result

    def __repr__(self):
        'Return a nicely formatted representation string'
        return 'Point(x=%r, y=%r)' % self

    def _asdict(self):
        'Return a new OrderedDict which maps
        field names to their values'
        return OrderedDict(zip(self._fields, self))

    def _replace(_self, **kwds):
        'Return a new Point object
        replacing specified fields with new values'
        result = _self._make(map(kwds.pop, ('x', 'y'), _self))
        if kwds:
            raise ValueError(
            'Got unexpected field names: %r' % kwds.keys())
        return result

    def __getnewargs__(self):
        'Return self as a plain tuple.
        Used by copy and pickle.'
        return tuple(self)

    __dict__ = _property(_asdict)

    def __getstate__(self):
        'Exclude the OrderedDict from pickling'
        pass

    x = _property(_itemgetter(0),
                  doc='Alias for field number 0')

    y = _property(_itemgetter(1),
                  doc='Alias for field number 1')


>>> p = Point(11, y=22)     # instantiate with 
...                         # positional or keyword arguments
>>> p[0] + p[1]             # indexable like 
...                         # the plain tuple (11, 22)
33
>>> x, y = p                # unpack like a regular tuple
>>> x, y
(11, 22)
>>> p.x + p.y               # fields also accessible by name
33
>>> p                       # readable __repr__ 
...                         # with a name=value style
Point(x=11, y=22)
\end{verbatim}