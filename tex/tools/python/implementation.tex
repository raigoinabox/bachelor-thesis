\subsubsection{Implementation}

A programming language by itself is not useful. It also needs an implementation.
An implementation is a program that evaluates the syntax of the program into
machine actions. There are 3 ways to create an implementation:
\begin{description}
  \item[Interpretation] Interpretation parses the source code and performs the
  instructions directly
  \item[Compilation] Compilation is transforming the current source code into a
  another format that can then be interpreted. Compiled language can also
  compile into another compiled language which will compile that into another
  language and so on.
  \item[just-in-time (JIT) compilation] JIT takes a hybrid approach of
  interpretation and compilation. While interpreting the program, the JIT
  compiler will observe what parts of the program are most often interpreted and
  will compile those parts.
\end{description}
Generally the compiling phase is called the compile time and the interpretation
phase is called the runtime. Compiled languages are usually considered fastest,
because compilation can heavily optimise the programmer's code, which leads to
a faster runtime. Interpretation however is faster and easier to use for the
programmer because the compilation step is skipped. For this reason interpreted
languages are usually used for scripting. JIT implementations are usually as easy to use as interpreted implementations, but
are a lot faster. However because the analysation is fairly complex and runs
parallel to the execution of the program, JIT compilers take a lot more
memory.\cite{website:jit-memory}

The official implementation of Python is CPython. Every formal change to the
language will be almost immediately mirrored in this implementation. It will be
supported for a long time will be the most current and up-to-date compared to
other implementations. CPython is a bytecode interpreter. It compiles to an
intermediate bytecode, which it then interprets. It compiles every time the
source file has been changed. This implementation is slow compared to
other languages\cite{website:python-speed}, but enables hooking the script to a
C module. Since the C programming language is fast,\cite{website:c-vs-python} it
is possible to write most of the program in Python and write the bottlenecks in
C.

The most popular alternative implementation to CPython is \pypy. \pypy{} is a JIT
compiler written in Python. It is popular because of its speed. As of \today,
\pypy{} is about 6.2~times faster than pure
CPython.\cite{website:python-pypy-speed} There are two disadvantages: \pypy{}
doesn't have hooks into C and PyPy isn't as up-to-date as CPython. As of \today,
Python 3 support for \pypy is in the beta state while CPython supports Python
3.4.0. Also since PyPy doesn't have hooks into C, it can't speed up the
bottlenecks of a program. With simple calculation from
\cite{website:c-vs-python,website:python-pypy-speed}, C gcc is still about 3
times faster than PyPy.

There are also Jython and IronPython. Jython compiles down to Java bytecode. It
allows the programmer to use libraries from Java in his project. IronPython is
the same principle, only it compiles to the .NET bytecode. That gives IronPython
projects access to .NET libraries.