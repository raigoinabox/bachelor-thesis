\section{Results}

The most significant result is the error rate of prediction. To calculate it,
when the program collects the pages to train the machine learning algorithm,
30\% of the pages are randomly selected and separated into a separate set. After the
training is complete, the algorithm is then asked to calculate the label of each
page in the test set. The calculated label and the known label are then compared
and it is possible to tell, whether the calculation was wrong or not.
If we add up all the wrong predictions, we get the error rate or the prediction
accuracy.

Since the low-quality pages are just random pages, they might have a bias in any
feature dimension. However, with big enough sets, the bias will average out.
Also, because the test pages are also randomly selected, there might be a bias
in them too. Therefore, the error rate can be varied. Table
\ref{table:training-results}, however, shows that the variance is generally
small.

\begin{table}
\begin{tabulary}{\textwidth}{CCCCCC}
& Training iterations & Tests & Wrong predictions & Error rate (\%) &
Runtime (seconds) \\
\cline{2-6}
& 766 & 68 & 1 & 1.5 & 26 \\
& 680 & 68 & 1 & 1.5 & 20 \\
& 653 & 68 & 3 & 4.4 & 15 \\
& 645 & 68 & 1 & 1.5 & 22 \\
& 785 & 68 & 2 & 2.9 & 23 \\
& 760 & 68 & 1 & 1.5 & 22 \\
\hline
Average & 715 & 68 & 1.50 & 2.2 & 21.3 \\
\hline
Standard Deviation & 57 & 0 & 0.76 & 1.1 & 3.3
\end{tabulary}
\caption{Training results}
\label{table:training-results}
\end{table}

Another type of result is the number of iterations of training. The training
algorithm gradient descent looks for the local minimum or maximum. It iterates
over the result of the last iteration and makes it more precise. It is possible
for gradient descent to iterate for a long time or forever, if the required
precision or the step between iterations is too high. If the number of
iterations reaches 10000, the algorithm is considered to never finish and throws
an error. Through trial and error the allowed error rate was fixed at 0.1\% and
the step size was fixed at 50\%.

The practical use of this algorithm is creating a list of high-quality articles
for the Vikipeedia editors.

Writing notes:

\begin{itemize}
  \item ``Most significant'' alguses feels wrong. Feels like ``In my opinion''
  ehk subjective.
  \item Esimene lõik, viimane lause: feels treating the reader like a child.
  \item Teine lõik, esimene lause. Kas ma sõnastan seda ikka õigesti. Neil
  peakski olema bias olema low-quality teoreetiliselt, aga praktiliselt tahan ma
  completely average page.
\end{itemize}

Notes:

\begin{itemize}
  \item Not going to mention data normlisation here? I figured it's an
  implementation detail and not a result.
\end{itemize}
