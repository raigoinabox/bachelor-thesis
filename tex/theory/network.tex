Client is process that requires some service. Server offers that service. The client and server communicate with HTTP. HTTP is a protocol in which the client sends the server a request and the server processes the request and sends a response. HTTP is purely plaintext. The HTTP request is divided into headers and content. The request has an URL in the header. URL stands for Uniform Resource Locator. It is a way for the client to request a certain page or other resource on the server. Another thing that the URL can contain are extra parameters. From those parameters the server can return the resource in a different form. Usually the request content is empty.

Extensible Markup Language (XML) is a protocol to describe data. It tries to nest data between descriptive tags. The tags can be nested and the tags can have attributes. The tags are not preset, so every user can design it's own way of presenting data. The syntax of a XML tag is \texttt{<tag attribute1=“value1“ attribute2=“value2“>data</tag>}. A tag can't have the same attribute with different values. data can be normal text or more tags. \texttt{<tag />} is shorthand for \texttt{<tag></tag>} for when the user doesn't have data to insert.

The HTTP response also splits into a header and content. Inside the content is
usually the requested data. A browser is an application that sends HTTP requests
to servers and parses and visualises the response to the user. To help with the
visualisation and interactiveness of the pages, HTML was created. HTML is an
Extensible Markup Language (XML) where the tags are focused on giving text some
form of context. For example, the a is a tag for a link and the \texttt{h1} is a
tag for a level 1 header. The browser also parses the HTML into visual cues and
interactions with the user.

HTML, however, lacks the ability change the HTML tags the user is currently
seeing. This means, the programmer is unable to interact with the user. For that
purpose there exists JavaScript (JS). JavaScript is a dynamic general-purpose
programming language. JavaScript works by registering a JavaScript function with an HTML
event, so that when that event is fired, that JavaScript function is run.
JavaScript interacts with the user by changing the current HTML the user is
seeing. Because JavaScript is a general-purpose programming language, unlike
HTML, any arbitrary calculation is possible. JavaScript uses Asynchronous
JavaScript and XML (AJAX) to send HTTP requests to outside servers. Despite the
name, the returned format doesn't have to be XML.

HTML pages are purely intended for browsers to parse and a human to see. For an
another application to get data from a HTML page, it has to web scrape. Web
scraping is observing beforehand how a web page is built and later filtering out
the necessary data from the HTML. Another way is for the server to offer an
Application Programming Interface (API). An API is a way for a program to get
data from the server with a HTTP requst in a more formal and machine-friendly
form. There are multiple formats for getting the data: JavaSrict Object Notation
(JSON), a Domain specific XML and so on.

With JSON the HTTP content is one legal JavaScript Object, which makes it easy
to read into JavaScript. Another advantage of it over XML is that there is basic
type checking. Javascript allows a value to have a few different types with
different notations: a string, a number, another object, an array; and the three
constant values \verb;true;, \verb;false;,
\verb;null;.\cite{website:json-guide,website:json}
