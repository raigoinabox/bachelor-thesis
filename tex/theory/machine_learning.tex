Machine learning is used when the programmer doesn't know all about the domain he is writing for. It is then necessary to have the machine learn by itself by some criteria. Machine learning generally divides into two categories: supervised learning and unsupervised learning. Supervised learning is when you have a known data set where for a known input there will be a known output. In unsupervised learning there is no such data set and the programmer is mostly looking for correlation between the inputs. Unsupervised learning is mostly used for data mining.

Logistic regression is a form of supervised learning which is used for classification. Our dataset $D$ will consist of the input of the model, a matrix of the the vectors $\mathbf{x_n}$ and the required outputs $y_n$ for the inputs. A input vector $\mathbf{x_n}$ will consist of the numerical features of the data prefixed with a 1 for the bias. $y_n$ can have only 2 values: 1 or -1. There are more complex forms of logistic regression that can handle more than two values for $y$ but is out of the scope of this paper. $N$ will be the size of our dataset.
\[ D = \{ (\mathbf{x_1}, y_1), (\mathbf{x_2}, y_2), \ldots, (\mathbf{x_N}, y_N)\} \]
\[ \mathbf{x_n} = [ \begin{array}{cccc} 1 & x_1 & \ldots & x_d\end{array} ]^T \]
Similar to another supervised learning algorithm linear regression, logistic regression is a linear model. „All linear models make use of a "signal" $s$ which is a linear combination of the input vector $\mathbf{x}$ components weighed by the corresponding components in a weight vector $\mathbf{w}$.“[19]
\[\mathbf{w} = [\begin{array}{cccc} w_0 & w_1 & \ldots & w_d \end{array} ]^T \]
\[s = w_0 + w_1 x_1 + \cdots + w_d x_d = \sum_{i=0}^d w_i x_i = \mathbf{w} \cdot \mathbf{x} = \mathbf{w}^T \mathbf{x}\]
Linear regression will use the signal directly as output, but logistic regression will pass the signal through a sigmoid or logistic function and treat thats output as the probability that $y=1$.
\[h(\mathbf{x}) = \theta(s)\]
\[\theta(s) = \frac{e^s}{1+e^s} = \frac{1}{1 + e^{-s}}\]
NOT YET PORTED ILLUSTRATION

As shown in the picture the logistic function is good for translating between linear values and probability, because the higher the linear value, the higher the probability of the output value being 1 and the lower the linear value the higher the probability of the output being -1. At input value 0, the probability of it being either value is 0.5.

„We say that the data is generated by a noisy target.“[19]
\[ P(y|\mathbf{x}) = \left\{
\begin{array}{ll}
f(\mathbf{x}) & \mbox{for } y = +1 \\
1 - f(\mathbf{x}) & \mbox{for } y = -1
\end{array} \right. \]
We want to learn a hypothesis $h(x)$ that best fits the above target according to some error function.
\[h(\mathbf{x}) = \theta \left( \mathbf{w}^T \mathbf{x} \right)\approx f(\mathbf{x})\]
„It's important to note that the data does not tell you the probability of a label but rather what label the sample has after being generated by the target distribution.“[19]. The goal of the training will be to calculate the weight vector w so that it minimizes some kind of in-sample error measure.
\[\mathbf{w}_h = \argmin_w E_{in} (\mathbf{w})\]
Our error measure will be based on likelihood. Likelihood is the probability of generating the data with a model. Likelihood will be high if the hypothesis is similar to the target distribution. \emph{TODO: Controversy? What? Millegipärast pidi likelihood olema controversial, aga ma ei saanud aru, miks.} Let's assume that the data was generated by the hypothesis:
\[P(y|\mathbf{x}) = \left\{ 
\begin{array}{ll}
h(\mathbf{x}) & \mbox{for } y = + 1 \\
1 - h(\mathbf{x}) & \mbox{for } y = -1
\end{array}
\right.\]
\[h(\mathbf{x}) = \theta (\mathbf{w}^T \mathbf{x})\]
Let's try to remove the cases using the property $\theta(-s) = 1 - \theta (s)$.
\[\left.
\begin{array}{l}
\mbox{if $y=+1$ then } h(\mathbf{x}) = \theta (\mathbf{w}^T \mathbf{x}) = \theta (y \mathbf{w}^T \mathbf{x}) \\
\mbox{if $y=-1$ then } 1 - h(\mathbf{x}) = 1 - \theta (\mathbf{w}^T \mathbf{x}) = \theta (-\mathbf{w}^T \mathbf{x}) = \theta (y \mathbf{w}^T \mathbf{x})
\end{array}
\right\} P(y | \mathbf{x}) = \theta(y \mathbf{w}^T \mathbf{x})\]
Let's denote an arbitrary hypothesis g, in which case the likelyhood is defined as:
\[L(D|g) = \prod_{n=1}^N P(y_n | \mathbf{x}_n) = \prod_{n=1}^N \theta(y_n \mathbf{w}_g^T \mathbf{x}_n)\]
To find the best hypothesis, we have to find the best weight vector $\mathbf{w}$.
\begin{eqnarray*}
\mathbf{w} & = & \argmax_\mathbf{w} L \left( D|h \right) = \argmax_\mathbf{w} \prod_{n=1}^N  \theta \left( y_n \mathbf{w}^T \mathbf{x}_n \right) = \argmax_\mathbf{w} \ln \left( \prod_{n=1}^N \theta\left( y_n \mathbf{w}^T \mathbf{x}_n \right) \right) \\
& = & \argmax_\mathbf{w} \frac{1}{N} \ln \left( \prod_{n=1}^N \theta \left( y_n \mathbf{w}^T \mathbf{x}_n \right) \right) = \argmin_\mathbf{w} \left[ - \frac{1}{N} \ln \left( \prod_{n=1}^N \theta \left( y_n \mathbf{w}^T \mathbf{x} \right) \right) \right] \\
& = & \argmin_\mathbf{w} \frac{1}{N} \sum_{n=1}^N \ln \left( \frac{1}{\theta \left( y_n \mathbf{w}^T \mathbf{x}_n \right)} \right) = \argmin_\mathbf{w} \frac{1}{N} \sum_{n=1}^N \ln \left( 1 + e^{-y_n \mathbf{w}^T \mathbf{x}_n} \right)
\end{eqnarray*}
We have derived a good form for the error measure, which is the loss function or the average point error.
\[ E_{in} \left( \mathbf{w} \right) = \frac{1}{N} \sum_{n=1}^N \ln \left( 1 + e^{-y_n \mathbf{x}_n \mathbf{w}^T} \right) = \frac{1}{N} \sum_{n=1}^N e \left( h \left( \mathbf{x}_n \right), y_n \right) \]
\[ e \left( h \left( \mathbf{x}_n \right), y_n \right) = \ln \left( 1 + e^{-y_n \mathbf{x}_n \mathbf{w}^T} \right) \]
We minimise the error function using gradient descent. Gradient descent works by moving the current value towards the local minimum. With the derivative, one can calculate the necessary direction and the rough distance of the local minimum from the current value. Therefore training works with the formula:
\[ \mathbf{w}_{i+1} = \mathbf{w}_i - \eta \nabla E_{in} \left( \mathbf{w}_i \right) \]
Where $\eta$ is the learning rate. We need the derivative of the point error function and the average point error.
\[ \frac{d}{d \mathbf{w}} e \left( h \left( \mathbf{x}_n \right), y_n \right) = \frac{-y_n \mathbf{x}_n e^{-y_n \mathbf{w}^T \mathbf{x}_n}}{1 + e^{-y_n \mathbf{w}^T \mathbf{x}_n}} = -\frac{y_n \mathbf{x}_n}{1 + e^{y_n \mathbf{w}^T \mathbf{x}_n}} \]
\begin{eqnarray*}
\nabla E_{in} \left( \mathbf{w} \right) & = & \frac{d}{d \mathbf{w}} \left[ \frac{1}{N} \sum_{n=1}^N e \left( h \left( \mathbf{x}_n \right), y_n \right) \right] = \frac{1}{N} \sum_{n=1}^N \frac{d}{d \mathbf{w}} e \left( h \left( \mathbf{x}_n \right), y_n \right) \\
& = & \frac{1}{N} \sum_{n=1}^N \left( -\frac{y_n \mathbf{x}_n}{1 + e^{y_n \mathbf{w}^T \mathbf{x}_n}} \right) = - \frac{1}{N} \sum_{n=1}^N \frac{y_n \mathbf{x}_n}{1 + e^{y_n \mathbf{w}^T \mathbf{x}_n}}
\end{eqnarray*}
\[ \mathbf{w}_{i+1} = \mathbf{w}_i - \eta \left( -\frac{1}{N} \sum_{n=1}^N \frac{y_n \mathbf{x}_n}{1 + e^{y_n \mathbf{w}_i^T \mathbf{x}_n}} \right) = \mathbf{w}_i + \eta \left( \frac{1}{N} \sum_{n=1}^N \frac{y_n \mathbf{x}_n}{1 + e^{y_n \mathbf{w}_i^T \mathbf{x}_n}} \right) \]
To lower the number of iterations, each feature of $\mathbf{x}$ should be normalised before it is used for predicting or training the model. Normalising means that the program calculates the standard score of each of the features which is then used instead. The standard score subtracts the mean from the features and divides that with the standard deviation of the values. A values standard score floats around the 0 value and roughly has the same absolute value as other standard scores. When normalising $\mathbf{x}$ for predicting, the mean and standard deviation cannot be enhanced with that data, because then the trained model will not be expecting such data. It would mean comparing two fundamentally different sets of data. It is important to note that the bias variable in $\mathbf{x}$ should not be normalised, because it will end in a divide by zero error.
