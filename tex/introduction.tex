\bulletsection{Introduction}
The Estonian Wikipedia has a lot of low-quality articles because
there are not enough editors. The aim of this thesis is to make the lives of the
editors easier by sorting the articles to high-quality and low-quality. The
editors can then focus more on the low-quality articles and less on the
high-quality articles.

Main questions I had to solve before writing the algorithm was:
\begin{itemize}
  \item What features of the article will weigh in calculating whether
  an article is high-quality or low-quality?
  \item What articles to label as high-quality and what articles to label as
  low-quality in machine learning?
  \item What is the most reasonable way to accomplish this task?
\end{itemize}

Accordingly, the work was divided into 5 steps:
\begin{enumerate}
  \item Researching the tools.
  \item Determining the features.
  \item Collecting the test data.
  \item Implementing the algorithm
  \item Validating the result.
\end{enumerate}

In the tools section, the prior work and tools used in creating this algorithm
will be described. The theoretical base for each tool and then the tool itself
will be described. Even though there are a lot of tools, the only choosable part
was the programming language. Python was selected because of prior experience
with the language, the current popularity and the availability of prior tools.

In the implementation section, the prerequisites, the interface, the
implementation of the algorithm and the infrastructure required and the
results will be described.
