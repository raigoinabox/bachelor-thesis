PyWikiBot is a Python framework which allows an application to communicate with
a MediaWiki instance through the MediaWiki API. Firstly the core component of
PyWikiBot is the Site object. The site object holds the connection to the
MediaWiki instance that PyWikiBot is supposed to query from. It is instantiated
without arguments. Instead it takes the instantiation data from the current
PyWikiBot configuration.

PyWikiBot is configurable with the \verb;user-config.py; file. On a linux system
the \verb;user-config.py; file has to be in the directory \verb;~/.pywikibot;.
The \verb;user-config.py; file is created with the script
\verb;generate_user_files.py;. It will ask the necessary questions and then
generate the \verb;user-config.py; file in the necessary location.

All bots in Wikipedia should have a user account that the bot is using to query
and make changes from. All those user accounts should also be flagged as bots,
so the admins can make better informed decisions in case of high load. PyWikiBot
follows this principle by not being able to run anonymously. When generating
user files, PyWikiBot will ask what will be the user name to run with. When
running the bot, PyWikiBot will also ask the user for the password. PyWikiBot
will remember the password and it is usually not necessary to write the password
in again even after a computer restart.

The site that PyWikiBot will query from is dependent on the family and the
language. These two are also asked when generating user files. A family is a
group of sites that are grouped together according to some theme. A language is
the the 2 character code of the language that the site was written in. The url
of the site doesn't have to be \verb;ll.family.tld;, where \verb;ll; is the
language and \verb;tld; is the top level domain of the site, like how it is with
wikipedia. Each site can have it's own url.

\subsubsection{Page}
The page object holds data about a single wiki page. A page instantation can
have 3 arguments. First is the source from which the page will be loaded. The
second is the title of the page. The third is the namespace from which the page
is loaded. The first argument is obligatory and the instantiating values depend
on it. If the source is an another page, then it will make a copy of the page
with the title overridden if it is given. If the source is a site, then it reads
the title and namespace argument and creates a link from them. If the source is
a link instance then it is remembered and the rest of the arguments are ignored.

A link represents a link to a page. A link has 3 instantiation arguments. The
first is the text of the link. The second is the site that the link is on. And
the third is the namespace to default to if the link text doesn't contain the
namespace where the link points to. The text is the only obligatory argument. If
no site is given then a new site is initialised. Unless otherwise specified the
default namespace will be the main namespace.

Another way to get pages is to use \verb;PageGenerator;s. A \verb;PageGenerator;
is a \verb;generator; that queries \verb;Page;s according to a specific
criteria. For example, the \verb;RandomPageGenerator; generates random pages
from the Wiki. It uses the MediaWiki generator API. There probably is a
PyWikiBot \verb;PageGenerator; for each MediaWiki generator.
