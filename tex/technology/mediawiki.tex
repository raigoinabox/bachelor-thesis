A wiki page is a page that owns a title in that wiki and is supposed to
aggregate information on the world wide web about the subject of the title. Most
of the content in the wiki pages are created or changed by the users of the
wiki. MediaWiki is software for a content-classification wiki. A content
classification wiki forbids writing personal knowledge and opinions on the wiki
page and only allows writing information that can be referenced. MediaWiki has a
separarate web page for discussion.\cite{wiki:mediawiki-about}

An interwiki link is a link that refers to another wiki page in the same wiki.
This way wikis don't have to duplicate information on different pages and just
refer to an another page. Links that link to outside the wiki are called
external links. Usually external links are in the page as references. There are
a special type of wiki pages that are called templates. They are not meant to
aggregate information, but that to act as a form template. Whenever a page links
to a template, it expands to form a part of the page it is linked on.

MediaWiki has 3 core dependencies. It is completely written in php, therefore it
needs php on the system it is running on. It also needs to run on a http server
like Apache that is also configured to have php enabled and index.php as a root
file. Running MediaWiki is an act of putting the MediaWiki directory in the http
server root directory and pointing the browser to the MediaWiki directory.
MediaWiki also needs a database service. Officially it supports MySQL 5.0.2+,
MariaDB 5.1+, PostgreSQL 8.1+, SQLite 3 and Oracle.\cite{wiki:mediawiki-installation-requirements}

The first time MediaWiki is run, the user is guided through an installation process which will install the database in the database service and create a config file that must be placed in the MediaWiki root directory.

MediaWiki also has an API that exposes the MediaWiki articles to bots without
the need to screen scrape. For that in the MediaWiki root directory is the file
api.php. Queries made at api.php with the right query parameters will return bot
friendly results.\cite{website:mediawiki-api} For example:

\url{http://en.wikipedia.org/w/api.php?format=json&action=query&titles=Main%20Page&prop=revisions&rvprop=content}

The format parameter tells what format the bot wants the data to be in. Each
format also has a version with a fm added to the end. The fm version
pretty-prints, what the bot would have seen, in HTML so it is easy for the
programmer to debug with the browser. Different possible formats include JSON,
XML, serialized PHP, WDDX and YAML, where xmlfm is the default. However, JSON is
the recommended format and all the other formats are deprecated.

The action parameter is the second required parameter. It tells what action the bot wants to do in the wiki. The rest of the query parameters are specific to the action. Most important to us is the query action, which is used to query data from wikipedia. The titles parameter is used to specify pages by name to query. Another way to specify pages is using a generator. Generators generate a list of pages. A bot can give a generator additional arguments by prefixing the parameters with the letter g. For example: \url{http://en.wikipedia.org/w/api.php?action=query&generator=allpages&gaplimit=3&gapfrom=Ba&prop=links|categories}

This example gets the links and categories of the first three pages. The \verb;generator; parameter tells the API which generator use to generate the pages. The \verb;gaplimit; tells the \verb;allpages; generator how many pages should the generator generate at one time. The \verb;gapfrom; tells what page the API should start listing from. The pages are alphabetically ordered. By default the \verb;allpages; generator uses the main namespace to generate pages from. Another significant generator is \verb;random; that generates pages randomly.

To limit a single user from putting a server on too much of a load. The API allows only a certain number of pages to be queried at once. To have bigger queries, the bot has to use continues. To let the server know, that the bot supports continues, the bot has to add the \verb;continue; parameter with an empty value to the query. When the query is big enough, the server returns a dictionary under a \verb;continue; parameter. The bot then takes the dictionary and queries again with the same arguments except the dictionary added as query parameters and arguments. The original \verb;continue; argument will be overwritten. If it is not possible to continue, the server will not return a continue parameter.
